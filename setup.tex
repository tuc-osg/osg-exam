\usepackage{fontspec}

\usepackage{polyglossia}
\setdefaultlanguage{german}

\usepackage[a4paper,vmargin={25mm,20mm},hmargin={20mm,25mm}]{geometry}
\usepackage{todonotes}			% definiert "\todo"
\usepackage{parskip}   			% keine Absatzeinrückung
\usepackage{tikz}	   			% Graphen malen
\usetikzlibrary{automata}

\usepackage{tabularx}
\usepackage{booktabs}
\usepackage{listings}
\lstset{basicstyle=\ttfamily,breaklines=true}
\usepackage{setspace}
\usepackage{xspace}
\usepackage{wrapfig}
\usepackage{amsmath}
\usepackage{multicol}
\usepackage{units}
\usepackage{paralist}
\usepackage{csquotes}
\usepackage{tabto}

% dicke Linien in Tabellen mit dem neuen \hlinewd Kommando
\usepackage{slashbox}
\makeatletter
\def\hlinewd#1{%
  \noalign{\ifnum0=`}\fi\hrule \@height #1 %
  \futurelet\reserved@a\@xhline}
\makeatother

% Frage und mögliche Antworten niemals über mehrere Seiten verteilen
\usepackage{etoolbox}
\AtBeginEnvironment{checkboxes}{\par\medskip\begin{minipage}{\linewidth}}\makeatletter
\AtEndEnvironment{checkboxes}{\if@correctchoice \endgroup \fi\end{minipage}}\makeatother

% Texte aus der Vorlage eindeutschen
\renewcommand{\solutiontitle}{}
\qformat{Aufgabe \thequestion \dotfill (\totalpoints\xspace Punkte)}
\bonusqformat{Aufgabe \thequestion \xspace(Bonus) \dotfill (\totalbonuspoints\xspace Punkte)}
\chqword{Aufgabe}
\chpword{Punkte}
\chtword{Gesamt}
\chsword{Erreicht}
\chbpword{Bonus}
\hqword{Aufgabe}
\hpword{Punkte}
\htword{Gesamt}
\hsword{Erreicht}
\marginbonuspointname{}

\header{\oddeven{Matrikelnr.: \studentid}{\thepage}}{}{\oddeven{\thepage}{\courseacronym \xspace \coursesemester}}      			  % Kopfzeile auf jeder Seite
\headrule									% Linie unter der Kopfzeile
\footer{}{}
\pointsinrightmargin   						% Der Name sagt alles
\shadedsolutions   							% der auch
