\documentclass[addpoints,answers]{exam}				% Für Lösungen "answers" als Option hinzufügen

\newcommand{\studentid}{Name}

\newcommand{\studentname}{Vorname Matrikelnummer}
	
% Check for options set through the Makefile
\ifdefined\genanswers
	\documentclass[addpoints,answers]{exam}
\else
	\documentclass[addpoints]{exam}
\fi

\ifdefined\studentid
\else
    \def\studentid{123456}
\fi

\ifdefined\studentname
\else
    \def\studentname{Max Mustermann}
\fi


\usepackage{fontspec}

\usepackage{polyglossia}
\setdefaultlanguage{german}

\usepackage[a4paper,vmargin={25mm,20mm},hmargin={20mm,25mm}]{geometry}
\usepackage{todonotes}			% definiert "\todo"
\usepackage{parskip}   			% keine Absatzeinrückung
\usepackage{tikz}	   			% Graphen malen
\usetikzlibrary{automata}

\usepackage{tabularx}
\usepackage{booktabs}
\usepackage{listings}
\lstset{basicstyle=\ttfamily,breaklines=true}
\usepackage{setspace}
\usepackage{xspace}

\usepackage{wrapfig}
\usepackage{amsmath}
\usepackage{multicol}
\usepackage{units}
\usepackage{paralist}
\usepackage{csquotes}
\usepackage{tabto}
\usepackage[framemethod=tikz]{mdframed}
\usepackage{array}
\usepackage{tabu}
\usepackage{enumitem}

% dicke Linien in Tabellen mit dem neuen \hlinewd Kommando
\usepackage{../lib/slashbox}
\makeatletter
\def\hlinewd#1{%
  \noalign{\ifnum0=`}\fi\hrule \@height #1 %
  \futurelet\reserved@a\@xhline}
\makeatother

% Frage und mögliche Antworten niemals über mehrere Seiten verteilen
\usepackage{etoolbox}
\AtBeginEnvironment{checkboxes}{\par\medskip\begin{minipage}{\linewidth}}\makeatletter
\AtEndEnvironment{checkboxes}{\if@correctchoice \endgroup \fi\end{minipage}}\makeatother

% Texte aus der Vorlage eindeutschen
\renewcommand{\solutiontitle}{}
\qformat{Aufgabe \thequestion \dotfill (\totalpoints\xspace Punkte)}
\bonusqformat{Aufgabe \thequestion \xspace(Bonus) \dotfill (\totalbonuspoints\xspace Punkte)}
\chqword{Aufgabe}
\chpword{Punkte}
\chtword{Gesamt}
\chsword{Erreicht}
\chbpword{Bonus}
\hqword{Aufgabe}
\hpword{Punkte}
\htword{Gesamt}
\hsword{Erreicht}
\marginbonuspointname{}

\header{\oddeven{Matrikelnr.: \studentid}{\thepage}}{}{\oddeven{\thepage}{\courseacronym \xspace \coursesemester}}      			  % Kopfzeile auf jeder Seite


\headrule									% Linie unter der Kopfzeile
\footer{}{}
\pointsinrightmargin   						% Der Name sagt alles
\shadedsolutions   							% der auch

% Fuer Klausuren in zwei Sprachen
\newcommand{\de}[1]{\textcolor{blue}{\footnotesize #1}}
\newcommand{\dep}[1]{\newline\de{#1}}

% Wasserzeichen bei Musterloesung
\ifprintanswers
\usepackage{draftwatermark}
\SetWatermarkText{Musterlösung}
\SetWatermarkColor[gray]{0.9}
\fi



% Fuer Klausuren in zwei Sprachen
\newcommand{\de}[1]{\textcolor{blue}{\footnotesize #1}}
\newcommand{\dep}[1]{\newline\de{#1}}

% Please note that the exam room is given in the Makefile, not here
\newcommand{\coursename}{Example}
\newcommand{\courseacronym}{Ex}
\newcommand{\coursesemester}{SS 17}


\begin{document}

\begin{tikzpicture}[overlay]
\node[align=left, draw=black, text width = 6.8cm,rounded corners, xshift=4.2cm, yshift=-3cm, minimum size=3.5cm, minimum width=8.5cm, inner sep=0pt] {
\texttt{\Huge \studentid}
\vspace{1em}
{\tiny
\\
\studentname
\\
\courseacronym \xspace (\coursesemester)
}
\vspace{1em}
\newline
\textcolor{red}{Nicht öffnen, warten Sie auf Anweisungen!}
};
\end{tikzpicture}

\begin{tikzpicture}[overlay]
\node[xshift=12.5cm, yshift=-2.3cm, align=center] {
\uppercase{
\huge Prüfung\\\\
\huge \courseacronym}};
\end{tikzpicture}


\vspace{5cm}


Willkommen bei der Prüfung für den Kurs {\textbf \coursename}.

\subsection*{Regeln}

\begin{itemize}
\item Sie haben zur Beantwortung aller Fragen 90 Minuten Zeit.
\item Durch Beginn der schriftlichen Bearbeitung bestätigen Sie, dass Sie gesundheitlich zu einer Prüfung in der Lage sind.
\item Bitte schreiben Sie deutlich. Wir können nur lesbare Antworten bewerten.
\item Als Hilfsmittel sind Schreibwerkzeuge und ein nicht-programmierbarer Taschenrechner gestattet.
\item Elektronische Geräte mit nichtflüchtigem Speicher oder Kommunikationsfähigkeiten - egal ob aktiviert oder nicht - sind untersagt. Das Vorhandensein solcher Geräte in Ihrer Reichweite kann als Betrugsversuch gewertet werden.
\item Gespräche mit anderen Studenten und der Austausch von schriftlichen Notizen sind während der Bearbeitungszeit untersagt und können als Betrugsversuch gewertet werden.
\item Wenn Sie eine Frage haben oder die Toilette besuchen müssen, melden Sie sich bitte und warten auf eine Aufsichtsperson.
\item Sie können zur Beantwortung auch einen Bleistift benutzen.
\end{itemize}

Frage 4 ist eine Bonusaufgabe und daher zum Erreichen der vollen Punktzahl nicht nötig.

Zum Bestehen der Prüfung müssen sie mindestens 0 Punkte erreichen. Viel Erfolg!
\\\\
\begin{center}
\combinedgradetable[h]
\end{center}
\pagebreak

\begin{questions}
%!TEX root = ../exam.tex

\question

All multiple choice questions have exactly one correct answer. If you mark more than one, you will get no points. If you need to correct yourself, invalidate all checkboxes of this particular question and write the word \enquote{CORRECT} beside your intended choice.
\dep{Alle Teilfragen haben jeweils genau eine korrekte Antwort. Kreuzen Sie diese an. Wenn Sie mehr als eine Antwort markieren, erhalten Sie keine Punkte. Bei Korrekturbedarf streichen Sie alle Kreise für die jeweilige Frage durch und schreiben das Wort \enquote{KORREKT} neben die angedachte Antwort.}

\begin{parts}

\part[1]
This is a question with...
\dep{Das ist eine Frage mit...}

\begin{checkboxes}
\correctchoice ... a correct answer.
\dep{... einer richtigen Antwort.}
\choice ... and a wrong answer.
\dep{... und einer falschen Antwort.}
\end{checkboxes}

\end{parts}



%\pagebreak
%!TEX root = ../exam.tex

\question[5]
  This is a question with just one part. We do not need the parts environment for that.
\dep{Diese Frage hat nur einen Teil. Deshalb brauchen wir die \enquote{parts} Umgebung nicht zu nutzen}

\begin{solution}
    Also the answer is really easy.
\end{solution}

\question
The following question is more complex and has sub-tasks. Therefore we use the parts environment.
\dep{Die folgende Frage ist komplexer mit mehreren Fragen. Dafür nutzen wir die \enquote{parts} Umgebung.}
\begin{parts}

\part[3]
Here we have some stuff which do not need to be printed in the solution sheet since the teacher does not need it (e.g. scheduling tables as a cushion).
\dep{Hier haben wir Dinge, welche wir in der Lösung nicht benötigen, z.B. Reservetabellen für Scheduling}

\ifprintanswers
\begin{solution}
Solution first part.
\end{solution}

\else
This is unnecessary for correction.
\fi

\part[3]
 Second part of the question.
 \dep{Zweiter Teil der Aufgabe}

\begin{solution}
Solution for the second part
\end{solution}

\end{parts}


%\pagebreak
%!TEX root = ../exam.tex

\bonusquestion[4]
This is a bonus question...the student does not need it to get full points.

\begin{solution}
And the answer to the question
\end{solution}

\end{questions}

%\clearpage
%\section*{Zusätzliche Notizen}
\end{document}